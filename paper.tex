\documentclass[11pt]{article}
\usepackage{amsmath, amssymb}
\usepackage{graphicx}

\usepackage{xeCJK}
\setmainfont{Times New Roman}
\setCJKmainfont{SimSun}

\usepackage{indentfirst}
\setlength{\parindent}{2em}

\usepackage{fancyhdr}
\pagestyle{fancy}
\usepackage{lastpage}
\rhead{Page \thepage{} of \pageref{LastPage}}
\cfoot{}
\renewcommand{\headrulewidth}{0pt}

\headheight 14pt
\begin{document}
\part{目录}

\part{摘要}

\part{假设}

\part{人口增长的简单模型}
\section{引述}
	就是抄一抄百科
\section{模型概述}
我们从人口的产生与消亡这两方面来看待人口的变动。人口的增长,一方面被人口的产生所促进,另一方面被人口的消亡所限制,所以是一个动态平衡的过程。一大波新出生的人口,在达到适育年龄的时候,会反过来对人口的增长起到促进作用;而当他们走向衰老乃至死亡的时候,又会加剧人口的减少。因此我们要使用科学的方法来看待这个问题。下面我们分别着手处理这个两个问题。
	\subsection{人口的衰老}
随着时间的流逝,一部分人口会衰老一岁,而剩下不幸的人则会因为各种各样的原因死去。其中死亡率是一个随着年龄而变化的数据,一般情况是婴儿的死亡率较高,然后随着年龄的增长而下降,到40、50岁时开始逐步回升。
	\subsection{人口的出生}
各个年龄层次的适育妇女,都有可能在考察的时间段内生育:对于个人而言,这是一个概率性事件;但是对于群体而言,生育事件的随机性就被大量的人群基数所磨灭,成为一个比率。因此,一段时间内婴儿的出生数目,等于平均生育率乘上适育妇女的人数。\\\\
\indent
但是,不同地区、学历的妇女在生产意愿上有着不同的态度。为了简化问题,我们认为地区与教育带给人的影响是近似独立的,因此可以直接对平均生育率进行修正。在这种假设下,一段时间内婴儿的出生人数,等于各年龄层的适育妇女乘上修正后的生育率之和。
\section{记号表}
见Table\ref{simple_symbol}。
	\begin{table}[]
		\centering
		\caption{人口增长的简单模型}
		\label{simple_symbol}
		\begin{tabular}{cccc}
			\hline
			符号		&	含义									&	相关的其他参数	&	备注 \\
			\hline
			$a$			&	年龄									&	& \\&&&$e_0$:小学及以下\\
			$e$			&	受教育水平(枚举量)	&									& 	$e_1$:中学	\\	&&&$e_2$:大学及以上	\\&&&$r_0$:城\\
			$r$			&	地区(枚举量)				&									& $r_1$:镇	\\	&&&$r_2$:乡	\\
			$t$			&	考察的时间段					&	& \\
			$I$			&	出生婴儿数						&	$e,r,t$		& \\
			$N$			&	总人口数							&	$a,e,r,t$ & \\
			$W$		&	女性人数							&	$a,e,r,t$	& \\
			$\beta$	&	生育率								&	$a,e,r$	& \\
			$\delta$	&	平均死亡率						&	$a$			& \\
			$\lambda$	&	修正系数						&	$a,e,r$	& \\
			$\mu_p$& 男女比例								&	& \\
			\hline
		\end{tabular} \\
	\end{table}
\section{数据引用}
		\begin{table}[]
		\centering
		\caption{来自第六次人口普查的数据}
		\label{simple_data}
		\begin{tabular}{c|ccc}
			\hline
			$I(r_k, e_l, 2010)$		&	$e_0$	&	$e_1$	&	$e_2$ \\
			\hline
			$r_0$							&		&		&	\\
			$r_1$							&		&		&	\\
			$r_2$							&		&		&	\\
			\hline
		\end{tabular}
	\end{table}
\section{方程的建立}
首先,考虑人口的自然衰老:
	\begin{equation}
		\label{simple_people_aging}
		N(a+1,t+1) = N(a,t) \cdot (1-\delta (a))
	\end{equation}
这里$\delta (a)$指的是$a$岁人口的平均死亡率。\\
\indent
再考虑所有地区,由受各种教育水平的妇女生育的婴儿数:
	\begin{equation}
		\label{simple_people_birth}
		N(0,t+1) = \sum_{r_k} \sum_{e_l} I(r_k, e_l, t)
	\end{equation}
这里$I(r_k, e_l, t+1)$表示在$t$这个考察时间段内,在$r_k$地区,受到$e_l$教育水平的妇女生育的婴儿数目。\\
\indent
考察$I(r_k, e_l, t)$是由各个不同年龄阶段的适龄妇女所生育的:
	\begin{equation}
		\label{simple_tiny_people_birth}
		I(r_k, e_l, t) = \sum_a \beta(a, r_k, e_l)W(a, r_k, e_l, t)
	\end{equation}
这里$\beta(a, r_k, e_l)$是$r_k$地区,受到$e_l$教育水平的妇女的生育率,$W(a, r_k, e_l, t)$是这段时间这类妇女的总人数。\\
\indent
总人口数与总女性数之间存在着简单的比例关系:
	\begin{equation}
		\label{simple_ratio_people_female}
		N(a,r,e,t) = W(a,r,e,t) \cdot (1+\mu_p)
	\end{equation}
这里$\mu_p$表示的是男性与女性人数的比例。\\
\indent
教育水平与地区对于生育率的修正,即$\beta(a, r_k, e_l)$,是这样构成的:
	\begin{equation}
		\label{simple_birthrate}
		\beta(a, r_k, e_l) = \overline{\beta(a)} \cdot \lambda_{r_k} \cdot \lambda_{e_l}
	\end{equation}
其中$\overline{\beta(a)}$是平均生育率,$\lambda_{r_k}$与$\lambda_{e_l}$分别是地区与教育水平分别对于生育率的修正;为了记号上的方便,令$\lambda_b$为生育修正参数矩阵, 其中
	\begin{equation}
		\label{simple_birthrate_factor}
		\lambda_b(k,l) = \lambda_{r_k} \cdot \lambda_{e_l}
	\end{equation}
\\
\indent
在进一步完善模型之前,我们给出$\lambda_b(k,l)$的计算公式与结果。事实上,根据等式\ref{simple_tiny_people_birth}与等式\ref{simple_birthrate},我们可以立即得到:
	\begin{equation}
		\label{simple_lambda_b_calc}
		\lambda_b(k,l) = \frac{I(r_k, e_l, t)}{\sum_a \overline{\beta(a)} W(a, r_k, e_l, t)}.
	\end{equation}
利用表\ref{simple_data}中的数据,代入$t=2010$,我们得到了$\lambda_b(k,l)$的矩阵形式:
	\begin{equation}
		\label{simple_lambda_b}
		\lambda_b(k,l) = \left|
		\begin{array}{ccc}
			??? & ??? & ??? \\
			??? & ??? & ??? \\
			??? & ??? & ???
		\end{array} \right|.
	\end{equation}
\section{小结}
	蛤
	
\part{人口增长模型修正与应用}
\section{假设}
	明确地列出哪些因素是要考虑的,而哪些还是不考虑
\section{记号表}
	\begin{table}[]
		\centering
		\caption{人口增长简单模型的修正}
		\label{amend_symbol}
		\begin{tabular}{cc}
			\hline
			符号		&	含义					\\
			\hline
			$I_i$		&	作为第$(i+1)$胎出生的婴儿数目	\\
			$W_i$		&	生育了$i$个孩子的妇女数目			\\
			\hline
		\end{tabular}
	\end{table}
\section{数据引用}
\section{模型的修正}
\subsection{依据生育情况区分妇女}
首先,适育妇女数,依照她们已生育孩子的数目,可分为$W_0$、$W_1$、$W_2$这三类,分别表示未生孩子、已生一个孩子与生了两个及以上的妇女数目。在这样的分类下,等式\ref{simple_tiny_people_birth}将被修正为如下带有下标的形式:
	\begin{equation}
		\label{amend_tiny_people_birth}
		I_i(r_k, e_l, t+1) = \sum_a \beta_i(a, r_k, e_l)W_i(a, r_k, e_l, t)
	\end{equation}
这里$I_i$是作为第$(i+1)$胎出生的婴儿数目,而$W_i$是生育过$i$胎的妇女数目。\\
\indent
同时,由于计划生育政策,已生育的妇女将会受到较大的政策阻力,因此她们的生育率将会减小,亦即:
	\begin{equation}
		\label{amend_birthrate}
		\beta_i(a, r_k, e_l) = \overline{\beta(a)} \lambda_i \cdot \lambda_b(k,l)
	\end{equation}
其中$\lambda_i$是已生$i$胎妇女受到政策阻力的修正因素。
\indent
在第一次修正后,中国人口增长趋势大致如图所示。

考虑到当二胎政策出台后,已生一胎的妇女生育意愿将会增大,因此等式\ref{simple_birthrate}中$\beta_1(a, r_k, e_l)$将会比原来大,被修正为:
	\begin{equation}
		\label{amend_birthrate_policydown}
		\beta(a, r_k, e_l) = \eta_a \cdot \overline{\beta(a)} \cdot \lambda_b(k,l)
	\end{equation}
其中$\eta_a$是$a$岁已生育一胎的妇女在政策放开后的相对生育意愿。

\section{数据预测}
	蛤
\section{小结}
	蛤
	
\part{把其他人的论文批判一番}
	\section{你们啊,naive}
		蛤
	\section{西方的人口模型我哪个没算过}
		蛤
	\section{开放二胎不知道比你们高到哪里去了}
		蛤
	\section{你问我支不支持计划生育,那我当然是支持的}
		蛤
	\section{我今天算是得罪了你们}
		蛤
	\section{小结}
		蛤
	
\part{北京市的人口增长模型}
	\section{讨论}
		蛤
	\section{模型的再次修正}
		蛤
	\section{模型的预测}
		蛤
	\section{小结}
		蛤
	
\part{政策对社会造成的影响}
	蛤

\part{养老金模型}
	蛤

\part{总结}
	蛤

\part*{参考文献}
	蛤

\end{document}
