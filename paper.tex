\documentclass[11pt]{article}
\usepackage{amsmath, amssymb}
\usepackage{graphicx}
\usepackage{xeCJK}
\setmainfont{Times New Roman}
\setCJKmainfont{SimSun}
\parindent 2em
\usepackage{fancyhdr}
\pagestyle{fancy}
\usepackage{lastpage}
\rhead{Page \thepage{} of \pageref{LastPage}}
\cfoot{}
\renewcommand{\headrulewidth}{0pt}
\headheight 14pt
\begin{document}
\part{目录}

\part{摘要}

\part{假设}

\part{人口增长的简单模型}
\section{引述}
	就是抄一抄百科
\section{模型概述}
我们从人口的产生与消亡这两方面来看待人口的变动。人口的增长,一方面被人口的产生所促进,另一方面被人口的消亡所限制,所以是一个动态平衡的过程。一大波新出生的人口,在达到适育年龄的时候,会反过来对人口的增长起到促进作用;而当他们走向衰老乃至死亡的时候,又会加剧人口的减少。因此我们要使用科学的方法来看待这个问题。下面我们分别着手处理这个两个问题:
\subsection{人口的衰老}
随着时间的流逝,一部分人口会衰老一岁,而剩下不幸的人则会因为各种各样的原因死去。其中死亡率是一个随着年龄而变化的数据,一般情况是婴儿的死亡率较高,然后随着年龄的增长而下降,到40、50岁时开始逐步回升。
\subsection{人口的出生}
各个年龄层次的适育妇女,都有可能在考察的时间段内生育:对于个人而言,
\section{记号表}
	列表
\section{方程的建立}
	列方程
\section{小结}
	蛤
	
\part{人口增长模型修正与应用}
\section{假设}
	明确地列出哪些因素是要考虑的,而哪些还是不考虑
\section{记号表}
	蛤
\section{模型的修正}
	蛤
\section{数据预测}
	蛤
\section{小结}
	蛤
	
\part{把其他人的论文批判一番}
	\section{你们啊,naive}
		蛤
	\section{西方的人口模型我哪个没算过}
		蛤
	\section{开放二胎不知道比你们高到哪里去了}
		蛤
	\section{你问我支不支持计划生育,那我当然是支持的}
		蛤
	\section{我今天算是得罪了你们}
		蛤
	\section{小结}
		蛤
	
\part{北京市的人口增长模型}
	\section{讨论}
		蛤
	\section{模型的再次修正}
		蛤
	\section{模型的预测}
		蛤
	\section{小结}
		蛤
	
\part{政策对社会造成的影响}
	蛤

\part{养老金模型}
	蛤

\part{总结}
	蛤

\part*{参考文献}
	蛤

\end{document}